%%%%%%%%%%%%%%%%%%%%%%%%%%%%%%%%%%%%%%%%%%%%%%%%%%%%%%%%%%%%%%%%%%%%%%%%%%%%%%%%
% Obxectivo: Lista de termos empregados no documento,                          %
%            xunto cos seus respectivos significados.                          %
%%%%%%%%%%%%%%%%%%%%%%%%%%%%%%%%%%%%%%%%%%%%%%%%%%%%%%%%%%%%%%%%%%%%%%%%%%%%%%%%

\newglossaryentry{bytecode}{
  name=bytecode,
  description={Código independente da máquina que xeran compiladores de determinadas linguaxes (Java, Erlang,\dots) e que é executado polo correspondente intérprete.}
}
\newglossaryentry{token}{
  name=token,
  description={Grupos de caracteres que representan la unidad fundamental del texto. Estos tokens son generados por un algoritmo tokenizador que segrega el texto en segmentos más pequeños siguiendo ciertas reglas, como espacios, signos de puntuación y caracteres especiales.}
}
\newglossaryentry{parquet}{
    name=parquet,
    description={Formato de almacenamiento de datos en columna que es especialmente eficiente para el análisis de grandes volúmenes de datos. Fue desarrollado originalmente por Cloudera y Twitter como parte del proyecto Apache Hadoop, y es ampliamente utilizado en entornos de procesamiento de datos y big data.}
}
\newglossaryentry{gradiente}{
    name=gradiente,
    description={Inclinación más marcada de la función de pérdida en relación a los parámetros del modelo. En un gráfico, el gradiente señalaría la dirección y velocidad con la que los parámetros del modelo deben ajustarse para llegar al mínimo de la función de pérdida, que representa el error más bajo.}
}
\newglossaryentry{QLoRA}{
    name=QLoRA,
    description={Técnica específica relacionada con la cuantificación de modelos de lenguaje preentrenados usando 4 bits. Este método se utiliza generalmente para reducir el tamaño y el consumo de recursos de un modelo de inteligencia artificial, especialmente en el ámbito del procesamiento de lenguaje natural.}
}
\newglossaryentry{few-shot learning}{
    name={few-shot learning},
    description={Técnica de aprendizaje automático donde el modelo es entrenado para realizar una tarea con una cantidad muy limitada de ejemplos de entrenamiento. Este enfoque es útil para situaciones donde la obtención de grandes cantidades de datos de entrenamiento es costosa o impráctica. Few-shot learning busca generalizar a partir de pocos ejemplos, permitiendo al modelo aprender y adaptarse rápidamente a nuevas tareas con mínima supervisión.}
}
\newglossaryentry{embeddings}{
    name={embeddings},
    description={Representaciones numéricas de datos, generalmente en forma de vectores, que capturan el significado y las relaciones semánticas de palabras, frases, oraciones o incluso documentos completos en un espacio vectorial continuo.}
}
\newglossaryentry{dataset}{
    name={dataset},
    description={Colección organizada de datos que generalmente se almacena y se accede electrónicamente desde un sistema informático.}
}
\newglossaryentry{adversaltraining}{
    name={adversarial training},
    description={Técnica de aprendizaje automático que tiene como objetivo hacer que los modelos sean más robustos contra ataques adversarios . Los ataques adversarios son intentos maliciosos de manipular o engañar a los modelos de aprendizaje automático realizando cambios pequeños e imperceptibles en los datos de entrada. Al incorporar ejemplos contradictorios, que son versiones modificadas de datos de entrada, durante el proceso de entrenamiento, los modelos pueden aprender a detectar y resistir mejor dichos ataques.}
}
