\chapter{Regulaciones en la Generación de Código}
\label{chap:estandares}

\lettrine{E}{}ste apéndice proporciona un resumen detallado de las certificaciones y estándares relevantes para la generación automática de código, enfocándose en cómo cada uno contribuye a asegurar la calidad, la seguridad y el cumplimiento legal de las soluciones desarrolladas.

\section{Estándares de Seguridad y Calidad}
\begin{itemize}
    \item \textbf{ISO/IEC 27001}: Estándar internacional que proporciona un marco para la gestión de la seguridad de la información. Es esencial para proteger los sistemas que soportan la generación automática de código contra amenazas y vulnerabilidades de seguridad. \cite{ISO27001}
    \item \textbf{ISO/IEC 25010}: Define un modelo de calidad para el software que incluye características clave como funcionalidad, fiabilidad, usabilidad y seguridad. Aplicable en la evaluación de la calidad del código generado automáticamente. \cite{ISO25010}
\end{itemize}

\section{Certificaciones de Cumplimiento de Privacidad}
\begin{itemize}
    \item \textbf{Certificación TrustArc}: Asegura que las prácticas de privacidad de una empresa cumplen con estándares internacionales como el \acrshort{GDPR}, garantizando el manejo adecuado de datos personales en aplicaciones de generación de código. \cite{TrustArc2024}
\end{itemize}

\section{Estándares de Desarrollo de Software}
\begin{itemize}
    \item \textbf{CMMI (Capability Maturity Model Integration)}: Modelo de madurez que mejora los procesos de desarrollo de software, asegurando que el software generado sea de alta calidad y desarrollado en un entorno controlado y sistemático. \cite{CMMIInstitute2024}
\end{itemize}

\section{Certificaciones Legales}
\begin{itemize}
    \item \textbf{ISO 31000}: Proporciona directrices para la gestión de riesgos asociados con el uso de tecnologías emergentes como la generación automática de código, ayudando a las organizaciones a identificar y mitigar posibles riesgos. \cite{ISO31000_2024}
\end{itemize}

\section{Estándares Específicos del Sector}
\begin{itemize}
    \item \textbf{PCI DSS}: Necesario para software que maneja transacciones de pago o datos de tarjetas de crédito, protegiendo contra fraudes y mejorando la seguridad en aplicaciones financieras. \cite{PCIDSS2024}
\end{itemize}

Cada uno de estos estándares y certificaciones representan un papel crucial en diferentes aspectos del desarrollo y despliegue de tecnologías de generación de código, desde la seguridad hasta el cumplimiento legal y la gestión de riesgos.


