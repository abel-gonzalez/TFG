\chapter{Conclusiones}
\label{chap:conclusions}

\lettrine{E}{}l estudio ha demostrado que la implementación de modelos de lenguaje masivos como \acrshort{GPT}, \acrshort{LLaMA}, y Mixtral puede mejorar significativamente la eficiencia en la generación automática de código. Los modelos entrenados han demostrado una capacidad notable para interpretar descripciones en lenguaje natural y generar código funcional correspondiente. A través de técnicas como Fine-Tuning y \acrfull{RAG}, se ha logrado ajustar estos modelos para optimizar su rendimiento y precisión en tareas específicas de generación de código.

\bigskip % Deja una línea en blanco

Los resultados de experimentos muestran que, aunque los modelos de \textbf{Fine-Tuning} son precisos en situaciones específicas, el enfoque \textbf{\acrshort{RAG}} es más flexible al combinar datos de múltiples fuentes, lo que podría mejorar la creatividad y adaptabilidad ante exigencias de programación complicadas. No obstante, se ha notado que también existen desafíos importantes en la gestión de recursos computacionales y en la optimización de estos modelos para tareas específicas, especialmente en cuanto al tiempo de entrenamiento y al uso de recursos.

\bigskip % Deja una línea en blanco

También se ha demostrado que no hay una única respuesta correcta al problema en general, es decir, dependiendo de las necesidades del proyecto, un tipo de enfoque y un modelo específico se adaptarán mejor. Es esencial contar con una comprensión general de los beneficios que ofrece cada tipo de modelo disponible en el mercado.

\bigskip % Deja una línea en blanco

Por último, este estudio busca iniciar una serie de investigaciones en el campo de los \acrshort{LLMs}, ya que las limitaciones de recursos y de tiempo han impedido profundizar en más modelos y abordar cada uno de ellos en mayor detalle.

\newpage
\section{Lecciones aprendidas y relación con la titulación}

La realización de este \acrshort{TFG} me ha permitido introducirme en el mundo de los \acrshort{LLMs} y de la \acrshort{GenAI}, donde, en gran parte, los conocimientos adquiridos en la asignatura de \textbf{Sistemas Inteligentes} me han servido como base para este nuevo mundo. De igual forma, el haber cursado la asignatura de \textbf{Programación Integrativa} me ha facilitado el manejo de \textit{datasets}, así como el uso de librerías como Tensorflow o Pandas para la manipulación de datos y \textit{datasets}.

\bigskip % Deja una línea en blanco

Además, he podido familiarizarme con modelos más allá de \acrshort{GPT} y así poder tener una visión general de esta tecnología tan innovadora que está teniendo tanto auge en este momento.

\bigskip % Deja una línea en blanco

Para adquirir los conocimientos mencionados a lo largo del proyecto, nos hemos enfrentado a diversas dificultades. Destacan la búsqueda del volumen adecuado de datos para entrenar los modelos, dado que nuestras capacidades computacionales y de almacenamiento son limitadas. Además, hemos tenido que asegurar la máxima similitud entre los entornos de prueba ajustando los parámetros de entrenamiento para mantener la coherencia en la investigación y resolver las particularidades de cada modelo a lo largo del proceso de entrenamiento. 

\bigskip % Deja una línea en blanco

En resumen, estos desafíos han enriquecido significativamente el proyecto, proporcionando una base sólida de experiencia y conocimiento en el campo de la \acrfull{GenAI}.

\section{Trabajo futuro}

La investigación llevada a cabo en este \acrfull{TFG} brinda una base sólida para investigaciones futuras sobre la aplicación de modelos de lenguaje masivo en la generación automatizada de código. 

\bigskip % Deja una línea en blanco

Una posible área de investigación futura es mejorar estos modelos para distintos lenguajes de programación. A pesar de los avances notables con Python, probar la flexibilidad y efectividad de estos modelos en otros idiomas, como JavaScript o Java, podría aumentar considerablemente su uso y importancia en diversos campos de la industria del software.

\bigskip % Deja una línea en blanco

Otra zona fundamental para investigaciones futuras es la creación de métricas y recursos para evaluar de forma eficaz la calidad del código producido. Esto no sólo aumentaría la confianza en la utilización de dichos modelos en ambientes de producción, sino también permitiría detectar y corregir falencias en cuanto a la claridad del código, rendimiento y seguridad.

\bigskip % Deja una línea en blanco

Por último, es crucial considerar las ramificaciones éticas y de seguridad que emergen al desarrollar código de manera automática. 