\chapter{Introducción}
\label{chap:introduccion}

\section{Contexto y motivación}

\lettrine{E}{}n la actual sociedad digital, el desarrollo y actualización de software se ha convertido en una pieza clave en todos los ámbitos de nuestra vida diaria, abarcando desde aplicaciones móviles hasta sistemas críticos en diversas industrias. Sin embargo, el proceso de desarrollo de software es a menudo complejo y demandante, requiriendo habilidades especializadas y una considerable inversión de tiempo. Frente a este desafío, la generación automática de código emerge como una herramienta prometedora para mejorar la eficiencia y la productividad en la creación de software.

\bigskip % Deja una línea en blanco

La generación automática de código utiliza \acrlong{LLMs}, como \acrshort{GPT}, \acrshort{LLaMA} y Mixtral, que han sido entrenados con vastos conjuntos de datos de texto. Estos modelos han demostrado una capacidad impresionante para comprender y generar texto coherente y relevante, aplicado especialmente en la programación. No obstante, la precisión y eficacia del código producido por estos modelos son aspectos esenciales, ya que un código erróneo o ineficaz puede resultar en fallos críticos y brechas de seguridad. Aunque tiene sus ventajas, la amplia adopción de esta tecnología plantea cuestiones éticas y prácticas importantes, que van desde la equidad y accesibilidad en la programación hasta la seguridad e integridad del software creado.

\section{Objetivos de la investigación}

Este estudio se sumerge en la exploración de la generación automática de código utilizando los \acrfull{LLMs}, enfocándose en desentrañar y comprender las profundas implicaciones éticas y sociales que emergen de esta avanzada tecnología. El objetivo central de la investigación es realizar un \textbf{análisis crítico de las herramientas y sistemas actuales que facilitan la generación automática de código}, identificando tanto los desafíos como las oportunidades que ello representa. Además, se busca evaluar los riesgos potenciales asociados y proponer metodologías innovadoras que no solo mitiguen estos riesgos sino que también promuevan un desarrollo tecnológico ético y responsable.

\bigskip % Deja una línea en blanco

El enfoque se dirige hacia una exploración integral que incluye desde la \textbf{revisión de la literatura} existente hasta la \textbf{evaluación empírica} de diversas técnicas y enfoques. Esta investigación tiene la intención de ofrecer una perspectiva balanceada que considere tanto la eficiencia técnica como la responsabilidad social, apuntando hacia la creación de un marco de trabajo que guíe la implementación segura y ética de la generación automática de código en diferentes contextos industriales y sociales.

\bigskip % Deja una línea en blanco

Los propósitos principales de este trabajo es:

\begin{itemize}
\item Realizar una \textbf{revisión exhaustiva de la documentación} relevante sobre la generación automática de código y los modelos de lenguaje avanzados, incluyendo consideraciones éticas.
\item Compilar y \textbf{organizar conjuntos de datos} para el entrenamiento, asegurando diversidad y calidad, para evaluar la capacidad de estos modelos en la generación de código.
\item \textbf{Entrenar y evaluar varios \acrlong{LLMs} }utilizando métricas como precisión, recall y F1-score.
\item \textbf{Comparar diferentes enfoques}  de generación de código, como el uso de modelos preentrenados y técnicas de Fine-Tuning, para evaluar su efectividad, rapidez y adaptabilidad a diversos escenarios de programación.
\item Desarrollar \textbf{estándares éticos} para la utilización de la generación automática de código en el sector del software, con el fin de maximizar sus beneficios y minimizar los riesgos.
\end{itemize}

\bigskip % Deja una línea en blanco

\section{Estructura del trabajo}

A continuación, se muestra una breve síntesis de la estructura general de la memoria, resaltando los distintos capítulos y los temas que se abordarán en cada uno:

\begin{itemize}
    \item En el \Cref{chap:estadodelarte}, se analizarán los desarrollos recientes y los avances tecnológicos en la generación automática de código y en los modelos de lenguaje masivos. Además, se explorarán las herramientas más conocidas que actualmente facilitan la generación de código.
    
    \item En el \Cref{chap:conceptos}, se detallarán los principios teóricos que respaldan los modelos de lenguaje a gran escala. En este capítulo se abordará el tema de las estructuras de redes neuronales, como los transformadores, métodos de aprendizaje profundo, y enfoques de ajuste fino y generación aumentada de recuperación.
    
    \item En el \Cref{chap:planificacion}, se explorará la metodología utilizada en el estudio, que implica la elaboración del proyecto a través de un diagrama de Gantt y la organización de las tareas mediante un tablero Kanban. Se abordará la metodología Scrum de desarrollo ágil, mejorando la coordinación y control del avance. Por último, se llevará a cabo una evaluación de los gastos vinculados, especificando los elementos humanos y materiales requeridos para realizar el proyecto.
    
    \item En el \Cref{chap:tecnologías}, se detallarán las tecnologías clave utilizadas en el proyecto, incluyendo el \textit{hardware}, como el equipo informático, y el \textit{software}, como Visual Studio Code y Google Colab. Se mencionarán también las \acrshort{API}s esenciales como las de OpenAI y Hugging Face, destacando su papel en el entrenamiento y evaluación de modelos de lenguaje masivos.
    
    \item En el \Cref{chap:desarrollo}, se presentará la implementación práctica de los modelos y la configuración experimental utilizada. Se comentarán los resultados obtenidos de los experimentos, proporcionando análisis y comparaciones detalladas sobre la efectividad de diferentes modelos y configuraciones en la generación de código.
    
    \item En el \Cref{chap:etica}, se abordarán las consideraciones de seguridad, los dilemas éticos y el cumplimiento normativo relacionados con la generación automática de código. Además, se tratarán los estándares éticos y legales actuales y cómo estos afectan el diseño y la implementación de sistemas de generación de código.
    
    \item En el \Cref{chap:conclusions}, se resumirán las conclusiones y se brindarán recomendaciones para investigaciones venideras. Se planteará la posibilidad de seguir mejorando la generación automática de código mediante avances en modelos de lenguaje masivos, señalando posibles áreas de desarrollo futuro.
\end{itemize}


